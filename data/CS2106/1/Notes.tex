\documentclass{article}
\begin{document}
\title{CS2106}

OS
D: Intermediary bw user and hardware

Mainframe
A: OS Types
N: CPU idle when IO
R: Multipro

Multiprogramming
A: OS

OS Advancements
A: OS

Time Sharing
A: OS Adv
D: Multiple users to interact with machine using terminals
T: Illusion of concurrency

Virtualisation
A: OS Adv
E: Every program executes as if it has all the resources to itself
E: OS gives the layer of abstraction

OS Motivation
A: OS

Abstraction
A: OS Mot
E: To standardise hardware configurations
E: Efficiency, Programmability, Portability

Resource Allocator
A: OS Mot
E: Multiple programs should be allowed to do simultaneously

Control Program
A: OS Mot
U: Malicious or Accidental use
U: Ensure isolation among users
X: Unclosed while loop

OS Design
A: OS
T: Robust, Flexible, Maintainable, Performant

Hardware
A: OS De

Software
A: OS De
R: OS

UI
A: Software

Kernel Mode
A: Software
E: Allows interaction with the /Hardware directly
R: OS

User Mode
A: Software
E: Cannot interact with /Hardware

Syscall Interface
A: User Mode
E: Allows user to interact with /Hardware via /OS
T: Cannot be used by /Kernel-Mode
T: Cannot use high level /Library

Library
A: User Mode
E: Interacts with /OS and /Hardware

OS Types
A: OS

Monolithic OS
A: OS Types
T: Kernel is one big program, with one /Syscall-Interface
T: Good SE principles are possible
B: Easy to call any function
N: Coupling, Complicated

Microkernel OS
A: OS Types
E: Minimum set of fnalities that needs to be implemented
E: All unimportant parts are in /User-Mode
B: If there is a bug in unimportant parts, OS can just kill it off
N: Bad performance, Needs to go through IPC

Virtual Machine
A: OS
E: Full control of the machine
E: Can run several OS on the same hardware
E: Software emulation of the hardware

Hypervisor
A: Virtual Machine
E: Manages and Creates the /Virtual-Machine
T: Below /OS

Hypervisor Types
A: Hypervisor

Bare Metal Hypervisor
A: Hypervisor Types
T: Runs directly on hardware

Type 2 Hypervisor
A: Hypervisor Types
E: Still have your own OS running, and guest OS runs inside the /Virtual-Machine

\end{document}